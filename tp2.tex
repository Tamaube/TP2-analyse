
\documentclass{article}

\usepackage[latin1]{inputenc}
\usepackage[T1]{fontenc}
\usepackage[francais]{babel}
\usepackage{graphicx}
\usepackage[top=2cm, bottom=2cm, left=2cm, right=2cm]{geometry}
\title{TP2}
\author{Benjamin Khenniche et Elisa Lescarret}
\date{9 novembre 2015}

\begin{document}

\part{TP2 Analyse - Rendu de Benjamin Khenniche et Elisa Lescarret}
\paragraph{}
\section{Exercie 1}
\subsection{Question 1}

\paragraph{On prend $x_0$ le milieu de $[a; b]$ c'est � dire $x_0 = 1/2 (a + b)$
$l$ se trouve donc dans un des deux intervalles $]a; x_0[$ ou $]x_0; b[$}

\paragraph{D'apr�s la m�thode par dichotomie si $f(a)f(x_0) < 0$ alors $l \in ]a; x_0[$ on pose alors $a_1 = a$ $b_1 = x0$}
\paragraph{Si $f(a)f(x_0) = 0$ alors $l = x_0$}
\paragraph{Si $f(a)f(x_0) > 0$ alors $l \in ]x_0; b[$ On pose $a_1 = x_0$ $b_1 = b$}

\paragraph{On pose ensuite $x_1 = 1/2(a_1 + b_1)$}
\paragraph{On construit donc une suite $x_n = 1/2(a_n + b_n)$ tel que $|x_n - l| \le \frac{(b -a)}{2^{n+1}}$}

\subsection{Question 2}

\paragraph{Soit $e_n$ l'�cart entre $x_n$ et $l$ On a donc $e_{n+1} = e_n/2$ testons si la m�thode est d'ordre 1}
\paragraph{On doit v�rifier que $|e_{n+1}|/|e_n|$ converge quand $n \rightarrow +\infty$}
					\[\lim\limits_{n \rightarrow +\infty}\frac{|e_{n + 1}|}{|e_n|} = \lim\limits_{n \rightarrow +\infty} \frac{|\frac{e_n}{2}|}{|e_n|} = \frac{1}{2}\]
\paragraph{On v�rifie pour l'ordre 2 si $\frac{|e_{n+1}|}{|e_n|^2}$ converge quand $n \rightarrow +\infty$}
					\[\lim\limits_{n \rightarrow +\infty} \frac{|e_{n+1}|}{|e_n|^2} = \lim\limits_{n \rightarrow +\infty} \frac{|e_{n}|}{2|e_n|^2} = \lim\limits_{n \rightarrow +\infty} \frac{1}{2|e_n|} = +\infty \]
\paragraph{$\frac{|e_{n+1}|}{|e_n|^2}$ diverge la m�thode est donc d'ordre 1}

\subsection{Question 3}
\paragraph{Pour commencer l'algorithme il faut que $f(a)$ et $f(b)$ soit de signe oppos�s pour que $0 \in [f(a), f(b)]$ donc $f(a)f(b) < 0$}
\paragraph{On a pour condition d'arr�t $|f(x_n)| \le \epsilon$}

\section{Exercice 2}
\subsection{Question 1}
\paragraph{On a $\frac{y - f(a)}{x - a} = \frac{f(b) - f(a)}{b - a} \Rightarrow g(a) = \frac{af(b) - bf(a)}{f(b) - f(a)}$}
\paragraph{On a donc}
\[x_{n+1} = g(x_n) = \frac{x_nf(b) - bf(x_n)}{f(b) - f(a)}\]

\subsection{Question 2}
\paragraph{$f(a)$ et $f(b)$ sont de signes oppos�s pour que $0 \in [f(a), f(b)]$ donc $f(a)f(b) < 0$}
\paragraph{Quand la convergence coupe l'axe en abscisse en $l$ cela correspond � $|x_{n + 1} - x_n| < \epsilon$}

\section{Exercice 3}
\subsection{Question 1}
\paragraph{Soit l'�quation $f(x) = x^3 -2x -5$ , nous cherchons les limites de $f(x)$}
\paragraph{$lim$ quand x-> +infini de $f(x)$  =  $+infini$}
\paragraph{$lim$ quand x-> -infini de $f(x)$  =  $-infini$}
\paragraph{$f'(x) = 3x� -2$}
\paragraph{TABLEAU DE VARIATION  : }
\paragraph{x : -\infty;  -racine(2/3); racine(2/3); +\infty}
\paragraph{$f'(x)$ : $+$; $-$ ; $+$}
\paragraph{$f(x)$ : -\infty; <0 ; <0; +\infty}
\paragraph{Il y a donc bien une seul solution. }

\subsection{Question 2}
\paragraph{Algo...}


\subsection{Question 3}
\paragraph{On a : $f(x) = x^3 -2x -5$ et $f'(x) = 3x-2$ et $f''(x) = 6x$}
\paragraph{Et nous cherchons un $u0 \in [a,b]$ tel que $f(u0) \times f''(u0) > 0$ et $ f(a) \times f(b) < 0$ }
\paragraph{Nous savons que $f(\sqrt{2/3}) < 0$ et $f(4) = 51$. choisissons donc l'interval $[\sqrt{2/3},4 ]$ et $u0 = 4$}


\section{Exercice 4}
\subsection{Question 1}

\paragraph{Comparaison avec Newton : Contrairement � la m�thode de Newton, la m�thode de la corde se sert de deux points ($x_0$ et $x_1$) pour calculer une "`tangente"' approximative. $x_n$ d�pend de $x_{n-1}$ et $x_{n-2}$. Deplus � l'initialisation, $x_0$ et $x_1$ n'ont pas besoin d'encadrer une racine de $f(x)$. Cependant, $f(x)$ doit �tre strictement monotone ($f'(x)\neq 0$) sur tout son domaine de d�finition.}

\paragraph{Comparaison avec Newton : La m�thode de la s�cante ressemble beaucoup � celle de Lagrange. A ceci pr�s que pour Lagrange, une des deux bornes (celle sup�rieure) est fix�e. Pour la s�cante, nous avont $x_0$ et $x_1$, l'une �tant la derni�re trouv�, l'autre datant d'une it�ration de plus.}


\subsection{Question 2}
\paragraph{}

\subsection{Question 3}
\paragraph{}


\end{document}