\documentclass{article}

\usepackage[latin1]{inputenc}
\usepackage[T1]{fontenc}
\usepackage[francais]{babel}
\usepackage{graphicx}
\usepackage[top=2cm, bottom=2cm, left=2cm, right=2cm]{geometry}
\title{TP2}
\author{Benjamin Khenniche et Elisa Lescarret}
\date{9 novembre 2015}

\begin{document}

\part{TP2 Analyse - Rendu de Benjamin Khenniche et Elisa Lescarret}
\paragraph{}
\section{Exercie 1}
\subsection{Question 1}

\paragraph{On prend $x_0$ le milieu de $[a; b]$ c'est � dire $x_0 = 1/2 (a + b)$
$l$ se trouve donc dans un des deux intervalles $]a; x_0[$ ou $]x_0; b[$}

\paragraph{D'apr�s la m�thode par dichotomie si $f(a)f(x_0) < 0$ alors $l \in ]a; x_0[$ on pose alors $a_1 = a$ $b_1 = x0$}
\paragraph{Si $f(a)f(x_0) = 0$ alors $l = x_0$}
\paragraph{Si $f(a)f(x_0) > 0$ alors $l \in ]x_0; b[$ On pose $a_1 = x_0$ $b_1 = b$}

\paragraph{On pose ensuite $x_1 = 1/2(a_1 + b_1)$}
\paragraph{On construit donc une suite $x_n = 1/2(a_n + b_n)$ tel que $|x_n - l| \le \frac{(b -a)}{2^{n+1}}$}

\subsection{Question 2}

\paragraph{Soit $e_n$ l'�cart entre $x_n$ et $l$ On a donc $e_{n+1} = e_n/2$ testons si la m�thode est d'ordre 1}
\paragraph{On doit v�rifier que $|e_{n+1}|/|e_n|$ converge quand $n \rightarrow +\infty$}
					\[\lim\limits_{n \rightarrow +\infty}\frac{|e_{n + 1}|}{|e_n|} = \lim\limits_{n \rightarrow +\infty} \frac{|\frac{e_n}{2}|}{|e_n|} = \frac{1}{2}\]
\paragraph{On v�rifie pour l'ordre 2 si $\frac{|e_{n+1}|}{|e_n|^2}$ converge quand $n \rightarrow +\infty$}
					\[\lim\limits_{n \rightarrow +\infty} \frac{|e_{n+1}|}{|e_n|^2} = \lim\limits_{n \rightarrow +\infty} \frac{|e_{n}|}{2|e_n|^2} = \lim\limits_{n \rightarrow +\infty} \frac{1}{2|e_n|} = +\infty \]
\paragraph{$\frac{|e_{n+1}|}{|e_n|^2}$ diverge la m�thode est donc d'ordre 1}

\subsection{Question 3}
\paragraph{Pour commencer l'algorithme il faut que $f(a)$ et $f(b)$ soit de signe oppos�s pour que $0 \in [f(a), f(b)]$ donc $f(a)f(b) < 0$}
\paragraph{On a pour condition d'arr�t $|f(x_n)| \le \epsilon$}

\section{Exercice 2}
\subsection{Question 1}
\paragraph{On a $\frac{y - f(a)}{x - a} = \frac{f(b) - f(a)}{b - a} \Rightarrow g(a) = \frac{af(b) - bf(a)}{f(b) - f(a)}$}
\paragraph{On a donc}
\[x_{n+1} = g(x_n) = \frac{x_nf(b) - bf(x_n)}{f(b) - f(a)}\]

\subsection{Question 2}
\paragraph{$f(a)$ et $f(b)$ sont de signes oppos�s pour que $0 \in [f(a), f(b)]$ donc $f(a)f(b) < 0$}
\paragraph{Quand la convergence coupe l'axe en abscisse en $l$ cela correspond � $|x_{n + 1} - x_n| < \epsilon$}
					

\end{document}